\documentclass{article}\usepackage[]{graphicx}\usepackage[]{color}
%% maxwidth is the original width if it is less than linewidth
%% otherwise use linewidth (to make sure the graphics do not exceed the margin)
\makeatletter
\def\maxwidth{ %
  \ifdim\Gin@nat@width>\linewidth
    \linewidth
  \else
    \Gin@nat@width
  \fi
}
\makeatother

\definecolor{fgcolor}{rgb}{0.345, 0.345, 0.345}
\newcommand{\hlnum}[1]{\textcolor[rgb]{0.686,0.059,0.569}{#1}}%
\newcommand{\hlstr}[1]{\textcolor[rgb]{0.192,0.494,0.8}{#1}}%
\newcommand{\hlcom}[1]{\textcolor[rgb]{0.678,0.584,0.686}{\textit{#1}}}%
\newcommand{\hlopt}[1]{\textcolor[rgb]{0,0,0}{#1}}%
\newcommand{\hlstd}[1]{\textcolor[rgb]{0.345,0.345,0.345}{#1}}%
\newcommand{\hlkwa}[1]{\textcolor[rgb]{0.161,0.373,0.58}{\textbf{#1}}}%
\newcommand{\hlkwb}[1]{\textcolor[rgb]{0.69,0.353,0.396}{#1}}%
\newcommand{\hlkwc}[1]{\textcolor[rgb]{0.333,0.667,0.333}{#1}}%
\newcommand{\hlkwd}[1]{\textcolor[rgb]{0.737,0.353,0.396}{\textbf{#1}}}%
\let\hlipl\hlkwb

\usepackage{framed}
\makeatletter
\newenvironment{kframe}{%
 \def\at@end@of@kframe{}%
 \ifinner\ifhmode%
  \def\at@end@of@kframe{\end{minipage}}%
  \begin{minipage}{\columnwidth}%
 \fi\fi%
 \def\FrameCommand##1{\hskip\@totalleftmargin \hskip-\fboxsep
 \colorbox{shadecolor}{##1}\hskip-\fboxsep
     % There is no \\@totalrightmargin, so:
     \hskip-\linewidth \hskip-\@totalleftmargin \hskip\columnwidth}%
 \MakeFramed {\advance\hsize-\width
   \@totalleftmargin\z@ \linewidth\hsize
   \@setminipage}}%
 {\par\unskip\endMakeFramed%
 \at@end@of@kframe}
\makeatother

\definecolor{shadecolor}{rgb}{.97, .97, .97}
\definecolor{messagecolor}{rgb}{0, 0, 0}
\definecolor{warningcolor}{rgb}{1, 0, 1}
\definecolor{errorcolor}{rgb}{1, 0, 0}
\newenvironment{knitrout}{}{} % an empty environment to be redefined in TeX

\usepackage{alltt}
\usepackage{natbib}
\usepackage[unicode=true]{hyperref}
\usepackage{geometry}
\geometry{tmargin=1in,bmargin=1in,lmargin=1in,rmargin=1in}


\IfFileExists{upquote.sty}{\usepackage{upquote}}{}
\begin{document} 
\title{STAT243-PS4}
\author{Jinhui Xu}
\date{September 2017}

\maketitle

\section{Other students}
I discuss some problems with Xin Shi.  

\section{Question 1}

\subsection{(a)}
There is only one copy. Because we can see that data and input share same address.
\begin{knitrout}
\definecolor{shadecolor}{rgb}{0.969, 0.969, 0.969}\color{fgcolor}\begin{kframe}
\begin{alltt}
\hlstd{x} \hlkwb{<-} \hlnum{1}\hlopt{:}\hlnum{10}
\hlstd{f} \hlkwb{<-} \hlkwa{function}\hlstd{(}\hlkwc{input}\hlstd{)\{}
  \hlkwd{print}\hlstd{(}\hlkwd{.Internal}\hlstd{(}\hlkwd{inspect}\hlstd{(input)))}       \hlcom{#check the address of input}
  \hlstd{data} \hlkwb{<-} \hlstd{input}
  \hlkwd{print}\hlstd{(}\hlkwd{.Internal}\hlstd{(}\hlkwd{inspect}\hlstd{(data)))}        \hlcom{#check the address of data}
        \hlstd{g} \hlkwb{<-} \hlkwa{function}\hlstd{(}\hlkwc{param}\hlstd{)} \hlkwd{return}\hlstd{(param} \hlopt{*} \hlstd{data)}
        \hlkwd{return}\hlstd{(g)}
\hlstd{\}}
\hlstd{data}\hlkwb{<-}\hlnum{100}
\hlstd{myFun} \hlkwb{<-} \hlkwd{f}\hlstd{(x)}
\end{alltt}
\begin{verbatim}
## @11745fe10 13 INTSXP g0c4 [NAM(2)] (len=10, tl=0) 1,2,3,4,5,...
##  [1]  1  2  3  4  5  6  7  8  9 10
## @11745fe10 13 INTSXP g0c4 [NAM(2)] (len=10, tl=0) 1,2,3,4,5,...
##  [1]  1  2  3  4  5  6  7  8  9 10
\end{verbatim}
\end{kframe}
\end{knitrout}

\subsection{(b)}
The size of the serialized object is doubled. The reason is that R store both input and data even though they have same address.
\begin{knitrout}
\definecolor{shadecolor}{rgb}{0.969, 0.969, 0.969}\color{fgcolor}\begin{kframe}
\begin{alltt}
\hlstd{x} \hlkwb{<-} \hlkwd{rnorm}\hlstd{(}\hlnum{1e5}\hlstd{)}
\hlstd{f} \hlkwb{<-} \hlkwa{function}\hlstd{(}\hlkwc{input}\hlstd{)\{}
  \hlstd{data} \hlkwb{<-} \hlstd{input}
        \hlstd{g} \hlkwb{<-} \hlkwa{function}\hlstd{(}\hlkwc{param}\hlstd{)} \hlkwd{return}\hlstd{(param} \hlopt{*} \hlstd{data)}
        \hlkwd{return}\hlstd{(g)}
\hlstd{\}}
\hlstd{myFun} \hlkwb{<-} \hlkwd{f}\hlstd{(x)}
\hlkwd{object.size}\hlstd{(x)}
\end{alltt}
\begin{verbatim}
## 800040 bytes
\end{verbatim}
\begin{alltt}
\hlkwd{length}\hlstd{(}\hlkwd{serialize}\hlstd{(myFun,}\hlkwa{NULL}\hlstd{))}
\end{alltt}
\begin{verbatim}
## [1] 1606454
\end{verbatim}
\end{kframe}
\end{knitrout}


\subsection{(c)}
When the function contains the command: data=input. myFun can get the value of data even x is removed.
However, when we delete that command, myFun need the value of input of f which means that myFun needs the value of x. So if we rm x, there would be an error.
\begin{knitrout}
\definecolor{shadecolor}{rgb}{0.969, 0.969, 0.969}\color{fgcolor}\begin{kframe}
\begin{alltt}
\hlstd{x} \hlkwb{<-} \hlnum{1}\hlopt{:}\hlnum{10}
\hlstd{f} \hlkwb{<-} \hlkwa{function}\hlstd{(}\hlkwc{data}\hlstd{)\{}
  \hlstd{g} \hlkwb{<-} \hlkwa{function}\hlstd{(}\hlkwc{param}\hlstd{)} \hlkwd{return}\hlstd{(param} \hlopt{*} \hlstd{data)}
  \hlkwd{return}\hlstd{(g)}
\hlstd{\}}
\hlstd{myFun} \hlkwb{<-} \hlkwd{f}\hlstd{(x)}
\hlkwd{rm}\hlstd{(x)}
\hlstd{data} \hlkwb{<-} \hlnum{100}
\hlkwd{myFun}\hlstd{(}\hlnum{3}\hlstd{)}
\end{alltt}


{\ttfamily\noindent\bfseries\color{errorcolor}{\#\# Error in myFun(3): 找不到对象'x'}}\begin{alltt}
\hlkwd{ls}\hlstd{(}\hlkwc{envir}\hlstd{=}\hlkwd{environment}\hlstd{(myFun))}
\end{alltt}
\begin{verbatim}
## [1] "data" "g"
\end{verbatim}
\end{kframe}
\end{knitrout}

\subsection{(d)}
We can use force to force the value of data.
\begin{knitrout}
\definecolor{shadecolor}{rgb}{0.969, 0.969, 0.969}\color{fgcolor}\begin{kframe}
\begin{alltt}
\hlstd{x} \hlkwb{<-} \hlnum{1}\hlopt{:}\hlnum{10}
\hlstd{f} \hlkwb{<-} \hlkwa{function}\hlstd{(}\hlkwc{data}\hlstd{)\{}
  \hlkwd{force}\hlstd{(data)}
  \hlstd{g} \hlkwb{<-} \hlkwa{function}\hlstd{(}\hlkwc{param}\hlstd{)} \hlkwd{return}\hlstd{(param} \hlopt{*} \hlstd{data)}
  \hlkwd{return}\hlstd{(g)}
\hlstd{\}}
\hlstd{myFun} \hlkwb{<-} \hlkwd{f}\hlstd{(x)}
\hlkwd{rm}\hlstd{(x)}
\hlstd{data} \hlkwb{<-} \hlnum{100}
\hlkwd{myFun}\hlstd{(}\hlnum{3}\hlstd{)}
\end{alltt}
\begin{verbatim}
##  [1]  3  6  9 12 15 18 21 24 27 30
\end{verbatim}
\end{kframe}
\end{knitrout}


\section{Question 2}
\subsection{(a)}
When change the a vector of list, I find that the address of the relevant changes and the other one does not change. Therefore, R would create a new vector.
\begin{knitrout}
\definecolor{shadecolor}{rgb}{0.969, 0.969, 0.969}\color{fgcolor}\begin{kframe}
\begin{alltt}
\hlstd{list1}\hlkwb{=}\hlkwd{list}\hlstd{(}\hlkwc{a}\hlstd{=}\hlkwd{rnorm}\hlstd{(}\hlnum{1e5}\hlstd{),}\hlkwc{b}\hlstd{=}\hlkwd{rnorm}\hlstd{(}\hlnum{1e5}\hlstd{))}
\hlkwd{.Internal}\hlstd{(}\hlkwd{inspect}\hlstd{(list1))}
\end{alltt}
\begin{verbatim}
## @116e802a0 19 VECSXP g0c2 [NAM(2),ATT] (len=2, tl=0)
##   @11301d000 14 REALSXP g0c7 [] (len=100000, tl=0) 0.260864,1.72475,-1.39273,-0.308712,0.459819,...
##   @109baf000 14 REALSXP g0c7 [] (len=100000, tl=0) 0.720782,-0.750896,-0.00154814,-0.591199,-0.465081,...
## ATTRIB:
##   @110353bf0 02 LISTSXP g0c0 [] 
##     TAG: @103824140 01 SYMSXP g1c0 [MARK,NAM(2),LCK,gp=0x6000] "names" (has value)
##     @116e802d8 16 STRSXP g0c2 [] (len=2, tl=0)
##       @103021b98 09 CHARSXP g1c1 [MARK,gp=0x61] [ASCII] [cached] "a"
##       @1032eab98 09 CHARSXP g1c1 [MARK,gp=0x61] [ASCII] [cached] "b"
\end{verbatim}
\begin{alltt}
\hlstd{list1[[}\hlnum{1}\hlstd{]][}\hlnum{1}\hlstd{]}\hlkwb{<-}\hlnum{100}
\hlkwd{.Internal}\hlstd{(}\hlkwd{inspect}\hlstd{(list2))}
\end{alltt}
\begin{verbatim}
## @11864b118 19 VECSXP g1c2 [MARK,NAM(2),ATT] (len=2, tl=0)
##   @10b47e000 14 REALSXP g1c7 [MARK,NAM(2)] (len=100000, tl=0) -1.05714,0.0947067,-1.0191,1.53438,-2.26516,...
##   @10b700000 14 REALSXP g1c7 [MARK,NAM(2)] (len=100000, tl=0) 0.19425,-0.332612,0.753014,0.510845,-0.115673,...
## ATTRIB:
##   @135938190 02 LISTSXP g1c0 [MARK] 
##     TAG: @103824140 01 SYMSXP g1c0 [MARK,NAM(2),LCK,gp=0x6000] "names" (has value)
##     @11864b188 16 STRSXP g1c2 [MARK,NAM(2)] (len=2, tl=0)
##       @103021b98 09 CHARSXP g1c1 [MARK,gp=0x61] [ASCII] [cached] "a"
##       @1032eab98 09 CHARSXP g1c1 [MARK,gp=0x61] [ASCII] [cached] "b"
\end{verbatim}
\end{kframe}
\end{knitrout}

\subsection{(b)}
According to the adddress of two lists before any change, we know that there is no copy-on-change.When the change is made, only the address of the relevant vector changes. Therefore, only a copy of the relevant vector is made.
\begin{knitrout}
\definecolor{shadecolor}{rgb}{0.969, 0.969, 0.969}\color{fgcolor}\begin{kframe}
\begin{alltt}
\hlstd{list2}\hlkwb{=}\hlkwd{list}\hlstd{(}\hlkwc{a}\hlstd{=}\hlkwd{rnorm}\hlstd{(}\hlnum{1e5}\hlstd{),}\hlkwc{b}\hlstd{=}\hlkwd{rnorm}\hlstd{(}\hlnum{1e5}\hlstd{))}
\hlstd{list2_cp}\hlkwb{=}\hlstd{list2}
\hlkwd{.Internal}\hlstd{(}\hlkwd{inspect}\hlstd{((list2)))}
\end{alltt}
\begin{verbatim}
## @116103c38 19 VECSXP g0c2 [NAM(2),ATT] (len=2, tl=0)
##   @110900000 14 REALSXP g0c7 [] (len=100000, tl=0) -0.315374,1.05064,1.03944,0.722292,-0.809419,...
##   @110d00000 14 REALSXP g0c7 [] (len=100000, tl=0) 0.0720925,0.810187,0.415644,-0.576821,3.24783,...
## ATTRIB:
##   @127c22808 02 LISTSXP g0c0 [] 
##     TAG: @103824140 01 SYMSXP g1c0 [MARK,NAM(2),LCK,gp=0x6000] "names" (has value)
##     @116103ca8 16 STRSXP g0c2 [] (len=2, tl=0)
##       @103021b98 09 CHARSXP g1c1 [MARK,gp=0x61] [ASCII] [cached] "a"
##       @1032eab98 09 CHARSXP g1c1 [MARK,gp=0x61] [ASCII] [cached] "b"
\end{verbatim}
\begin{alltt}
\hlkwd{.Internal}\hlstd{(}\hlkwd{inspect}\hlstd{((list2_cp)))}
\end{alltt}
\begin{verbatim}
## @116103c38 19 VECSXP g0c2 [NAM(2),ATT] (len=2, tl=0)
##   @110900000 14 REALSXP g0c7 [] (len=100000, tl=0) -0.315374,1.05064,1.03944,0.722292,-0.809419,...
##   @110d00000 14 REALSXP g0c7 [] (len=100000, tl=0) 0.0720925,0.810187,0.415644,-0.576821,3.24783,...
## ATTRIB:
##   @127c22808 02 LISTSXP g0c0 [] 
##     TAG: @103824140 01 SYMSXP g1c0 [MARK,NAM(2),LCK,gp=0x6000] "names" (has value)
##     @116103ca8 16 STRSXP g0c2 [] (len=2, tl=0)
##       @103021b98 09 CHARSXP g1c1 [MARK,gp=0x61] [ASCII] [cached] "a"
##       @1032eab98 09 CHARSXP g1c1 [MARK,gp=0x61] [ASCII] [cached] "b"
\end{verbatim}
\begin{alltt}
\hlcom{#the address of list2_cp is same with that of list2}
\hlstd{list2_cp[[}\hlnum{1}\hlstd{]][}\hlnum{1}\hlstd{]}\hlkwb{<-}\hlnum{100}
\hlkwd{.Internal}\hlstd{(}\hlkwd{inspect}\hlstd{(list2_cp))}
\end{alltt}
\begin{verbatim}
## @1163d1460 19 VECSXP g0c2 [NAM(1),ATT] (len=2, tl=0)
##   @110dc4000 14 REALSXP g0c7 [] (len=100000, tl=0) 100,1.05064,1.03944,0.722292,-0.809419,...
##   @110d00000 14 REALSXP g0c7 [NAM(2)] (len=100000, tl=0) 0.0720925,0.810187,0.415644,-0.576821,3.24783,...
## ATTRIB:
##   @119702d58 02 LISTSXP g0c0 [] 
##     TAG: @103824140 01 SYMSXP g1c0 [MARK,NAM(2),LCK,gp=0x6000] "names" (has value)
##     @116103ca8 16 STRSXP g0c2 [NAM(2)] (len=2, tl=0)
##       @103021b98 09 CHARSXP g1c1 [MARK,gp=0x61] [ASCII] [cached] "a"
##       @1032eab98 09 CHARSXP g1c1 [MARK,gp=0x61] [ASCII] [cached] "b"
\end{verbatim}
\begin{alltt}
\hlcom{#after change, we can find only the address of relevant vector changes}
\end{alltt}
\end{kframe}
\end{knitrout}

\subsection{(c)}
Notice the change of address after adding a vector into the second list. The address of two lists becomes different, but two original vectors still have original addresses. So the only change is that the second list creates a new vector while other vectors still share original addresses.
\begin{knitrout}
\definecolor{shadecolor}{rgb}{0.969, 0.969, 0.969}\color{fgcolor}\begin{kframe}
\begin{alltt}
\hlstd{list3}\hlkwb{=}\hlkwd{list}\hlstd{(}\hlkwc{a}\hlstd{=}\hlkwd{list}\hlstd{(}\hlkwd{rnorm}\hlstd{(}\hlnum{1e5}\hlstd{)),}\hlkwc{b}\hlstd{=}\hlkwd{list}\hlstd{(}\hlkwd{rnorm}\hlstd{(}\hlnum{1e5}\hlstd{)))}
\hlstd{list3_cp}\hlkwb{=}\hlstd{list3}
\hlkwd{.Internal}\hlstd{(}\hlkwd{inspect}\hlstd{(list3))}
\end{alltt}
\begin{verbatim}
## @1336c3940 19 VECSXP g0c2 [NAM(2),ATT] (len=2, tl=0)
##   @127ff1af8 19 VECSXP g0c1 [] (len=1, tl=0)
##     @109900000 14 REALSXP g0c7 [] (len=100000, tl=0) -0.902474,0.657533,-1.1857,-0.844848,0.079948,...
##   @127ff1b28 19 VECSXP g0c1 [] (len=1, tl=0)
##     @11301d000 14 REALSXP g0c7 [] (len=100000, tl=0) -0.789741,0.481347,-0.815168,-0.121418,1.07905,...
## ATTRIB:
##   @133723230 02 LISTSXP g0c0 [] 
##     TAG: @103824140 01 SYMSXP g1c0 [MARK,NAM(2),LCK,gp=0x6000] "names" (has value)
##     @1336c2a40 16 STRSXP g0c2 [] (len=2, tl=0)
##       @103021b98 09 CHARSXP g1c1 [MARK,gp=0x61] [ASCII] [cached] "a"
##       @1032eab98 09 CHARSXP g1c1 [MARK,gp=0x61] [ASCII] [cached] "b"
\end{verbatim}
\begin{alltt}
\hlkwd{.Internal}\hlstd{(}\hlkwd{inspect}\hlstd{(list3_cp))}
\end{alltt}
\begin{verbatim}
## @1336c3940 19 VECSXP g0c2 [NAM(2),ATT] (len=2, tl=0)
##   @127ff1af8 19 VECSXP g0c1 [] (len=1, tl=0)
##     @109900000 14 REALSXP g0c7 [] (len=100000, tl=0) -0.902474,0.657533,-1.1857,-0.844848,0.079948,...
##   @127ff1b28 19 VECSXP g0c1 [] (len=1, tl=0)
##     @11301d000 14 REALSXP g0c7 [] (len=100000, tl=0) -0.789741,0.481347,-0.815168,-0.121418,1.07905,...
## ATTRIB:
##   @133723230 02 LISTSXP g0c0 [] 
##     TAG: @103824140 01 SYMSXP g1c0 [MARK,NAM(2),LCK,gp=0x6000] "names" (has value)
##     @1336c2a40 16 STRSXP g0c2 [] (len=2, tl=0)
##       @103021b98 09 CHARSXP g1c1 [MARK,gp=0x61] [ASCII] [cached] "a"
##       @1032eab98 09 CHARSXP g1c1 [MARK,gp=0x61] [ASCII] [cached] "b"
\end{verbatim}
\begin{alltt}
\hlstd{list3_cp}\hlopt{$}\hlstd{b[[}\hlnum{2}\hlstd{]]}\hlkwb{<-}\hlkwd{rnorm}\hlstd{(}\hlnum{1e5}\hlstd{)}
\hlkwd{.Internal}\hlstd{(}\hlkwd{inspect}\hlstd{(list3_cp))}
\end{alltt}
\begin{verbatim}
## @1336bdb58 19 VECSXP g0c2 [NAM(1),ATT] (len=2, tl=0)
##   @127ff1af8 19 VECSXP g0c1 [NAM(2)] (len=1, tl=0)
##     @109900000 14 REALSXP g0c7 [] (len=100000, tl=0) -0.902474,0.657533,-1.1857,-0.844848,0.079948,...
##   @1336bdb90 19 VECSXP g0c2 [] (len=2, tl=0)
##     @11301d000 14 REALSXP g0c7 [NAM(2)] (len=100000, tl=0) -0.789741,0.481347,-0.815168,-0.121418,1.07905,...
##     @110e88000 14 REALSXP g0c7 [NAM(2)] (len=100000, tl=0) -0.311266,0.221599,0.132626,-0.370482,0.297938,...
## ATTRIB:
##   @1335a5440 02 LISTSXP g0c0 [] 
##     TAG: @103824140 01 SYMSXP g1c0 [MARK,NAM(2),LCK,gp=0x6000] "names" (has value)
##     @1336c2a40 16 STRSXP g0c2 [NAM(2)] (len=2, tl=0)
##       @103021b98 09 CHARSXP g1c1 [MARK,gp=0x61] [ASCII] [cached] "a"
##       @1032eab98 09 CHARSXP g1c1 [MARK,gp=0x61] [ASCII] [cached] "b"
\end{verbatim}
\end{kframe}
\end{knitrout}

\subsection{(d)}
Object.size is twice large as the result of gc. I guess that it is because two elements of list is stored in the same address, but object.size estimates the size of list equels to sum of size of each element.
\begin{knitrout}
\definecolor{shadecolor}{rgb}{0.969, 0.969, 0.969}\color{fgcolor}\begin{kframe}
\begin{alltt}
\hlkwd{gc}\hlstd{()}
\end{alltt}
\begin{verbatim}
##            used  (Mb) gc trigger  (Mb)  max used  (Mb)
## Ncells  1650699  88.2    2637877 140.9   2637877 140.9
## Vcells 30261474 230.9   97834876 746.5 109412705 834.8
\end{verbatim}
\begin{alltt}
\hlstd{tmp} \hlkwb{<-} \hlkwd{list}\hlstd{()}
\hlstd{x} \hlkwb{<-} \hlkwd{rnorm}\hlstd{(}\hlnum{1e7}\hlstd{)}
\hlstd{tmp[[}\hlnum{1}\hlstd{]]} \hlkwb{<-} \hlstd{x}
\hlstd{tmp[[}\hlnum{2}\hlstd{]]} \hlkwb{<-} \hlstd{x}
\hlkwd{.Internal}\hlstd{(}\hlkwd{inspect}\hlstd{(tmp))}
\end{alltt}
\begin{verbatim}
## @119390740 19 VECSXP g0c2 [NAM(1)] (len=2, tl=0)
##   @143800000 14 REALSXP g0c7 [NAM(2)] (len=10000000, tl=0) -0.0151081,-0.752749,-0.294608,2.24558,0.947043,...
##   @143800000 14 REALSXP g0c7 [NAM(2)] (len=10000000, tl=0) -0.0151081,-0.752749,-0.294608,2.24558,0.947043,...
\end{verbatim}
\begin{alltt}
\hlkwd{object.size}\hlstd{(tmp)}
\end{alltt}
\begin{verbatim}
## 160000136 bytes
\end{verbatim}
\begin{alltt}
\hlkwd{gc}\hlstd{()}
\end{alltt}
\begin{verbatim}
##            used  (Mb) gc trigger  (Mb)  max used  (Mb)
## Ncells  1650748  88.2    2637877 140.9   2637877 140.9
## Vcells 30261575 230.9   97834876 746.5 109412705 834.8
\end{verbatim}
\end{kframe}
\end{knitrout}

\section{Question 3}
Notice that in the original code,firstly, the if else is not necessary at all. So I directly calculate q without if else. Secondly I replace three nested for loops with simple computation of vector
\begin{knitrout}
\definecolor{shadecolor}{rgb}{0.969, 0.969, 0.969}\color{fgcolor}\begin{kframe}
\begin{alltt}
\hlstd{ll} \hlkwb{<-} \hlkwa{function}\hlstd{(}\hlkwc{Theta}\hlstd{,} \hlkwc{A}\hlstd{) \{}
  \hlstd{sum.ind} \hlkwb{<-} \hlkwd{which}\hlstd{(A}\hlopt{==}\hlnum{1}\hlstd{,} \hlkwc{arr.ind}\hlstd{=T)}
  \hlstd{logLik} \hlkwb{<-} \hlkwd{sum}\hlstd{(}\hlkwd{log}\hlstd{(Theta[sum.ind]))} \hlopt{-} \hlkwd{sum}\hlstd{(Theta)}
  \hlkwd{return}\hlstd{(logLik)}
\hlstd{\}}
\hlcom{#######################################}
\hlcom{##original code########################}
\hlstd{oneUpdate} \hlkwb{<-} \hlkwa{function}\hlstd{(}\hlkwc{A}\hlstd{,} \hlkwc{n}\hlstd{,} \hlkwc{K}\hlstd{,} \hlkwc{theta.old}\hlstd{,} \hlkwc{thresh} \hlstd{=} \hlnum{0.1}\hlstd{) \{}
  \hlstd{theta.old1} \hlkwb{<-} \hlstd{theta.old}
  \hlstd{Theta.old} \hlkwb{<-} \hlstd{theta.old} \hlopt \hlkwd{t}\hlstd{(theta.old)}
  \hlstd{L.old} \hlkwb{<-} \hlkwd{ll}\hlstd{(Theta.old, A)}
  \hlstd{q} \hlkwb{<-} \hlkwd{array}\hlstd{(}\hlnum{0}\hlstd{,} \hlkwc{dim} \hlstd{=} \hlkwd{c}\hlstd{(n, n, K))}
\hlcom{##############the following part would be revised}
  \hlkwa{for} \hlstd{(i} \hlkwa{in} \hlnum{1}\hlopt{:}\hlstd{n) \{}
    \hlkwa{for} \hlstd{(j} \hlkwa{in} \hlnum{1}\hlopt{:}\hlstd{n) \{}
      \hlkwa{for} \hlstd{(z} \hlkwa{in} \hlnum{1}\hlopt{:}\hlstd{K) \{}
        \hlkwa{if} \hlstd{(theta.old[i, z]}\hlopt{*}\hlstd{theta.old[j, z]} \hlopt{==} \hlnum{0}\hlstd{)\{}
        \hlstd{q[i, j, z]} \hlkwb{<-} \hlnum{0} \hlstd{\}} \hlkwa{else} \hlstd{\{}
           \hlstd{q[i, j, z]} \hlkwb{<-} \hlstd{theta.old[i, z]}\hlopt{*}\hlstd{theta.old[j, z]} \hlopt{/}\hlstd{Theta.old[i, j]}
         \hlstd{\}}
       \hlstd{\}}
     \hlstd{\}}
  \hlstd{\}}
\hlcom{################}
  \hlstd{theta.new} \hlkwb{<-} \hlstd{theta.old}
  \hlkwa{for} \hlstd{(z} \hlkwa{in} \hlnum{1}\hlopt{:}\hlstd{K) \{}
  \hlstd{theta.new[,z]} \hlkwb{<-} \hlkwd{rowSums}\hlstd{(A}\hlopt{*}\hlstd{q[,,z])}\hlopt{/}\hlkwd{sqrt}\hlstd{(}\hlkwd{sum}\hlstd{(A}\hlopt{*}\hlstd{q[,,z]))}
  \hlstd{\}}
  \hlstd{Theta.new} \hlkwb{<-} \hlstd{theta.new} \hlopt \hlkwd{t}\hlstd{(theta.new)}
  \hlstd{L.new} \hlkwb{<-} \hlkwd{ll}\hlstd{(Theta.new, A)}
      \hlstd{converge.check} \hlkwb{<-} \hlkwd{abs}\hlstd{(L.new} \hlopt{-} \hlstd{L.old)} \hlopt{<} \hlstd{thresh}
  \hlstd{theta.new} \hlkwb{<-} \hlstd{theta.new}\hlopt{/}\hlkwd{rowSums}\hlstd{(theta.new)}
  \hlkwd{return}\hlstd{(}\hlkwd{list}\hlstd{(}\hlkwc{theta} \hlstd{= theta.new,} \hlkwc{loglik} \hlstd{= L.new,}\hlkwc{converged} \hlstd{= converge.check))}
\hlstd{\}}

\hlcom{#######################################}
\hlcom{##revised code#########################}
\hlstd{oneUpdate_new} \hlkwb{<-} \hlkwa{function}\hlstd{(}\hlkwc{A}\hlstd{,} \hlkwc{n}\hlstd{,} \hlkwc{K}\hlstd{,} \hlkwc{theta.old}\hlstd{,} \hlkwc{thresh} \hlstd{=} \hlnum{0.1}\hlstd{) \{}
  \hlstd{theta.old1} \hlkwb{<-} \hlstd{theta.old}
  \hlstd{Theta.old} \hlkwb{<-} \hlstd{theta.old} \hlopt \hlkwd{t}\hlstd{(theta.old)}
  \hlstd{L.old} \hlkwb{<-} \hlkwd{ll}\hlstd{(Theta.old, A)}
  \hlstd{q} \hlkwb{<-} \hlkwd{array}\hlstd{(}\hlnum{0}\hlstd{,} \hlkwc{dim} \hlstd{=} \hlkwd{c}\hlstd{(n, n, K))}
\hlcom{####################begin.revised part}
  \hlkwa{for} \hlstd{(z} \hlkwa{in} \hlnum{1}\hlopt{:}\hlstd{K) \{}
       \hlstd{q[ , , z]} \hlkwb{<-} \hlstd{theta.old[, z]}\hlopt\hlkwd{t}\hlstd{(theta.old[ , z])} \hlopt{/}\hlstd{Theta.old}
  \hlstd{\}}
\hlcom{####################end.revised part}
  \hlstd{theta.new} \hlkwb{<-} \hlstd{theta.old}
  \hlkwa{for} \hlstd{(z} \hlkwa{in} \hlnum{1}\hlopt{:}\hlstd{K) \{}
  \hlstd{theta.new[,z]} \hlkwb{<-} \hlkwd{rowSums}\hlstd{(A}\hlopt{*}\hlstd{q[,,z])}\hlopt{/}\hlkwd{sqrt}\hlstd{(}\hlkwd{sum}\hlstd{(A}\hlopt{*}\hlstd{q[,,z]))}
  \hlstd{\}}
  \hlstd{Theta.new} \hlkwb{<-} \hlstd{theta.new} \hlopt \hlkwd{t}\hlstd{(theta.new)}
  \hlstd{L.new} \hlkwb{<-} \hlkwd{ll}\hlstd{(Theta.new, A)}
      \hlstd{converge.check} \hlkwb{<-} \hlkwd{abs}\hlstd{(L.new} \hlopt{-} \hlstd{L.old)} \hlopt{<} \hlstd{thresh}
  \hlstd{theta.new} \hlkwb{<-} \hlstd{theta.new}\hlopt{/}\hlkwd{rowSums}\hlstd{(theta.new)}
  \hlkwd{return}\hlstd{(}\hlkwd{list}\hlstd{(}\hlkwc{theta} \hlstd{= theta.new,} \hlkwc{loglik} \hlstd{= L.new,}\hlkwc{converged} \hlstd{= converge.check))}
\hlstd{\}}
\hlcom{# initialize the parameters at random starting values}
\hlstd{temp} \hlkwb{<-} \hlkwd{matrix}\hlstd{(}\hlkwd{runif}\hlstd{(n}\hlopt{*}\hlstd{K), n, K)}
\hlstd{theta.init} \hlkwb{<-} \hlstd{temp}\hlopt{/}\hlkwd{rowSums}\hlstd{(temp)}
\hlcom{#compare the time used }
\hlkwd{system.time}\hlstd{(out} \hlkwb{<-} \hlkwd{oneUpdate}\hlstd{(A, n, K, theta.init))}
\end{alltt}
\begin{verbatim}
##    user  system elapsed 
##   5.458   0.227   5.759
\end{verbatim}
\begin{alltt}
\hlkwd{system.time}\hlstd{(out_new} \hlkwb{<-} \hlkwd{oneUpdate_new}\hlstd{(A, n, K, theta.init))}
\end{alltt}
\begin{verbatim}
##    user  system elapsed 
##   0.736   0.335   1.075
\end{verbatim}
\begin{alltt}
\hlkwd{all.equal}\hlstd{(out,out_new)}
\end{alltt}
\begin{verbatim}
## [1] TRUE
\end{verbatim}
\end{kframe}
\end{knitrout}
The results are same while the time used by revised code decreases.

\section{Question 4}
Notice that in the function FYKD, the for loop is aim to generate vector x. However, the algorithm only need the first k value of vector x. So we can only calculate that part rather than entire vector.
\begin{knitrout}
\definecolor{shadecolor}{rgb}{0.969, 0.969, 0.969}\color{fgcolor}\begin{kframe}
\begin{alltt}
\hlstd{PIKK} \hlkwb{<-} \hlkwa{function}\hlstd{(}\hlkwc{x}\hlstd{,} \hlkwc{k}\hlstd{) \{}
\hlstd{x[}\hlkwd{sort}\hlstd{(}\hlkwd{runif}\hlstd{(}\hlkwd{length}\hlstd{(x)),} \hlkwc{index.return} \hlstd{=} \hlnum{TRUE}\hlstd{)}\hlopt{$}\hlstd{ix[}\hlnum{1}\hlopt{:}\hlstd{k]]}
\hlstd{\}}
\hlstd{FYKD} \hlkwb{<-} \hlkwa{function}\hlstd{(}\hlkwc{x}\hlstd{,} \hlkwc{k}\hlstd{) \{}
  \hlstd{n} \hlkwb{<-} \hlkwd{length}\hlstd{(x)}
\hlcom{#in the original code , the following code is to generate entire n values in vector x}
  \hlkwa{for}\hlstd{(i} \hlkwa{in} \hlnum{1}\hlopt{:}\hlstd{n) \{}
     \hlstd{j} \hlkwb{=} \hlkwd{sample}\hlstd{(i}\hlopt{:}\hlstd{n,} \hlnum{1}\hlstd{)}
     \hlstd{tmp} \hlkwb{<-} \hlstd{x[i]}
     \hlstd{x[i]} \hlkwb{<-} \hlstd{x[j]}
     \hlstd{x[j]} \hlkwb{<-} \hlstd{tmp}
  \hlstd{\}}
\hlkwd{return}\hlstd{(x[}\hlnum{1}\hlopt{:}\hlstd{k])}  \hlcom{# while we only need first k values }
\hlstd{\}}

\hlcom{#so revised code do not calculate the latter part of the vector}
\hlstd{FYKD_new} \hlkwb{<-} \hlkwa{function}\hlstd{(}\hlkwc{x}\hlstd{,} \hlkwc{k}\hlstd{) \{}
  \hlstd{n} \hlkwb{<-} \hlkwd{length}\hlstd{(x)}
  \hlkwa{for}\hlstd{(i} \hlkwa{in} \hlnum{1}\hlopt{:}\hlstd{k) \{}
     \hlstd{j} \hlkwb{=} \hlkwd{sample}\hlstd{(i}\hlopt{:}\hlstd{n,} \hlnum{1}\hlstd{)}
     \hlstd{tmp} \hlkwb{<-} \hlstd{x[i]}
     \hlstd{x[i]} \hlkwb{<-} \hlstd{x[j]}
     \hlstd{x[j]} \hlkwb{<-} \hlstd{tmp}
  \hlstd{\}}
\hlkwd{return}\hlstd{(x[}\hlnum{1}\hlopt{:}\hlstd{k])}
\hlstd{\}}
\end{alltt}
\end{kframe}
\end{knitrout}

\end{document}
